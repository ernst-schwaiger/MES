\section{IoT Security}
\label{IOTSecurity}
This section provides details on how the firmware update is secured.

\subsection{Secure Firmware Update Over-The-Air (SFOTA)}
Before it is possible to have a SFOTA process, it is important to define on what grounds e.g. network layout and cryptographic primitives the update is based upon. 

\paragraph{Network:}
The network will be a locally run IPv6 network created with the support of RIOT-OS' tool "Ethernet over Serial Driver" (ethos).
The tools helps creating a network instance with a custom interface name and a defined prefix.
In our case this project's configuration may look like this:
\begin{itemize}
  \item  Interface: \textbf{riot0}
  \item  Prefix: \textbf{2001:db8::/64}
\end{itemize}

\textbf{Note:} Although, the devices are serially connected to the device hosting the IPv6 network this way, the update still remains over-the-air as IPv6 packets are exchanged.

\paragraph{Firmware:}
The firmware must encrypted and signed on the firmware management server and later decrypted by the host devices that process and execute the update.
The encryption is done with existing implementations of RIOT-OS' Crypto library.
% This is my suggestion to not overcomplicate this and justification is also there
In this case, the algorithm chosen is CBC due to its simplicity and provides confidentiality.
AEAD ciphers were considered, however, in this case, it is not necessary as integrity is provided by the signature.
The signature algorithm used is Ed25519 since RIOT-OS provides the crypthographic library \textit{C25519}.

How the key is generated and distributed to be used for encryption/decryption as well as the firmware signature is described in \ref{KeyManagement} 

\paragraph{Process}
To update the firmware, the SUIT procedure is used where the manifest and the encrypted and signed firmware are uploaded onto the firmware management server. The build process is explained in \ref{FirmwareManagement}.
The hosts uses the manifest to execute the necessary steps to flash the device with the new firmware image.
Fetching the necessary data such as the SUIT manifest or the encrypted firmware relies on CoAP with the library nanoCoaP.

// TODO SUIT flowchart with devices