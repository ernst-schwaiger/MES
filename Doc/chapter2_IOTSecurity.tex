\section{IoT Security}
\label{IOTSecurity}
This section provides details on how the firmware update is secured.

\subsection{Secure Firmware Update Over-The-Air (SFOTA)}
Before it is possible to have a SFOTA process, it is important to define on what grounds e.g. network layout and cryptographic primitives the update is based upon. 

\paragraph{Network:}
The network will be a locally run IPv6 network created with the support of RIOT-OS' tool "Ethernet over Serial Driver" (ethos).
The tools helps creating a network instance with a custom interface name and a defined prefix.
In our case this project's configuration may look like this:
\begin{itemize}
  \item  Interface: \textbf{riot0}
  \item  Prefix: \textbf{2001:db8::/64}
\end{itemize}

\textbf{Note:} Although, the devices are serially connected to the device hosting the IPv6 network this way, the update still remains over-the-air as IPv6 packets are exchanged.

\paragraph{Encryption and signature:}
The firmware must be encrypted and signed on the firmware management server and later decrypted by the client devices that process and execute the update.
The encryption and signature for the firmware on the management server is done via a Python script with the library PyCryptodome\footnote{https://pycryptodome.readthedocs.io/en/latest/src/introduction.html (Accessed 16-12-2025)}.

The decryption and signature on the client nodes is implemented using the existing implementations of RIOT-OS' Crypto library.
% This is my suggestion to not overcomplicate this and justification is also there
Due to its simplicity, the algorithm AES-CBC is chosen. AEAD ciphers were considered, however, in this case, it is not necessary as integrity is provided by signatures.

The signature algorithm used is Ed25519 since both RIOT-OS and PyCryptodome support it.
Table \ref*{table:data-transfer} shows what data is exchanged from the management server and the client nodes and which are either plaintext, encrypted and/or signed.

\paragraph{SUIT:}
To update the firmware, the SUIT procedure is used where the signed manifest and the encrypted and signed firmware are uploaded in the firmware management server.
The server triggers a notification and the client nodes begin the update procedure by fetching the manifest first.

The client nodes use the manifest to execute the necessary steps to fetch the image, validate, decrypt and finally flash the device with the new firmware image.
Fetching the necessary data such as the SUIT manifest or the encrypted firmware relies on CoAP with the library nanoCoaP.

The SUIT manifest will be extended by adding steps to decrypt the firmware and session AES key as well as validations. 
The encrypted session key resides in the SUIT manifest. 

Figure \ref{fig:suit} shows how an overview how the update procedure is done via notification.


\begin{table}[h!]
  \centering
  \begin{tabular}{|l|l|}
  \hline
   \textbf{Data sent to client} & \textbf{Protected} \\
   \hline
   Trigger notification by FW management server & Plaintext \\
   \hline
   SUIT Manifest & Plaintext, Signed \\
   \hline
   Firmware image & Encrypted, Signed \\
   \hline
   Session AES key (Within Manifest) & Encrypted, Signed \\
  \hline
  \end{tabular}
  \caption{Data transferred between FW Management server and Client nodes}
  \label{table:data-transfer}
\end{table}

\begin{figure}[h]
	\centering
	\includegraphics[width=0.8\textwidth]{images/SUIT-MES.drawio.png}
	\caption{Overview - SFOTA process with SUIT}
	\label{fig:suit}
\end{figure}

\begin{table}[h!]
  \centering
  \begin{tabular}{|c|l|l|}
  \hline
  \textbf{Category} & \textbf{Main Risk} & \textbf{Mitigation} \\
  \hline
  Spoofing & Spoofed update server & Signed SUIT manifest \\
  \hline
  Tampering & Modified SUIT manifest & Signed SUIT manifest \\
  \hline
   & Modified firmware image & End-to-end encryption \\
  \hline
  Repudiation & Server denies update action & Audit logs \\
  \hline
   & Device denies update action & Audit logs \\
  \hline
  Info disclosure & SSH interception & Strong SSH keys, host verification \\
  \hline
  Denial of Service & Fake update notifications & Not handled \\
  \hline
  Elevation of privilege & Root Certificate compromise & Not handled \\
  \hline
  \end{tabular}
  \caption{STRIDE Threat Model}
  \label{table:stride}
\end{table}

\paragraph{STRIDE:}
Potential threats need to be considered when implementing this update process. Therefore, table \ref{table:stride} presents some main risks across the STRIDE categories and mitigations that are done in the implementation. 
However, some risks were identified that are not going to be handled due to complexity and time limitations.

