\section{Project description}
\label{ProjectDescription}

This section provides a brief overview of the project, covering the basic idea with simple sketches for visualization.

\subsection{Problem description}

\paragraph{}
The aim of the project is to develop a system which allows to distribute newer versions of a piece of firmware onto a number of client nodes. The firmware update is created on a \emph{Build System}, then transferred to a \emph{Firmware Management Server}, which subsequently sends packages containing the updated firmware to a number of \emph{Client} nodes which are installing the updates.

\paragraph{}
Figure \ref{fig:system} displays an overview of the system. A self-signed \emph{Root/CA} certificate is used for signing a \emph{Build\_Signing} certificate which is stored together with the private key on a crypto token.  When a new firmware update package is built, it is signed using that key to prove its authenticity to the \emph{Firmware Management Server} which is receiving that package via ssh/scp.

\begin{figure}[h]
	\centering
	\includegraphics[width=0.8\textwidth]{images/Overview.drawio}
	\caption{System Description}
	\label{fig:system}
\end{figure}

\paragraph{}
A service on the \emph{Firmware Management Server} detects a newly arrived firmware update, verifies the validity of the \emph{Build\_Signing} certificate and the validity of the signature. The service puts the received firmware into a SUIT update package enclosing all required meta data, signs the package using its private signing key (of the \emph{Mgmnt\_Signing} certificate), then generates a session AES key, which it uses to encrypt the SUIT update package. For each \emph{Client} node it then encrypts the AES key using the \emph{Client} public key and stores it along with the package signature.

\paragraph{}
The encrypted SUIT update package, along with the encrypted AES key and the SUIT signature is then sent to all \emph{Client} nodes via \emph{6LoWPAN}.

\paragraph{}
On the client system, both the encrypted firmware and the encrypted key and signature arrive. The client verifies the validity of the \emph{Mgmnt\_Signing} certificate, then the signature of the received package using the certificate. If the verification was successful, the client first decrypts the AES key using its private key, then it decrypts the package using the AES key. In the next step, the client can install/flash the firmware update.

\paragraph{}
For the handling of certificates that either will expire soon or have been compromised, a second process takes care of replacing/revoking them on the respective Build System and \emph{Client} nodes.
